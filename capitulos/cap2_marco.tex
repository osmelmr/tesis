\chapter{Marco teórico / Estado del arte}

\section{Fundamentos de blockchain}

La tecnología \textit{blockchain} puede definirse como un libro mayor digital, distribuido, compartido e inmutable. De acuerdo con IBM, ``blockchain es un libro de contabilidad digital compartido e inmutable, que permite el registro de transacciones y el seguimiento de activos dentro de una red empresarial''. En esencia, funciona como una base de datos descentralizada replicada en múltiples nodos, lo que la hace resistente a la manipulación o pérdida de información.

Cada bloque contiene un hash del bloque anterior, formando una cadena que preserva la integridad de los datos. Este diseño basado en criptografía y mecanismos de consenso (como \textit{proof of work} o \textit{proof of stake}) garantiza que las transacciones solo sean validadas cuando la mayoría de la red está de acuerdo, evitando modificaciones arbitrarias por parte de un único actor. Gracias a estas propiedades, ``no se puede eliminar ninguna transacción, ni siquiera por parte de un administrador del sistema''.

Entre los beneficios principales de blockchain destacan la transparencia, trazabilidad e inmutabilidad. Todos los participantes pueden observar el registro completo de transacciones, conformando un ``registro a prueba de manipulaciones'' donde cada entrada posee una auditoría verificable. IBM señala que blockchain provee ``seguridad, transparencia y confianza sin depender de intermediarios'', además de brindar ``trazabilidad instantánea con un registro de auditoría transparente''.

Asimismo, la Guía de Referencia de Blockchain del Gobierno de Colombia destaca que esta tecnología permite a múltiples usuarios acordar un registro inmutable sin autoridad central, lo que contribuye a sistemas más seguros y confiables. En resumen, blockchain integra descentralización, consenso criptográfico, inmutabilidad y contratos inteligentes, características esenciales para aplicaciones donde la integridad y auditabilidad son fundamentales.

\section{Votación electrónica y blockchain}

El voto electrónico comprende sistemas donde los ciudadanos emiten su voto mediante tecnologías digitales, ya sea mediante máquinas electrónicas en urnas físicas o mediante modalidades remotas como voto por Internet, SMS o correo electrónico. La Comisión Europea y diversas fuentes resaltan ventajas relevantes: mayor accesibilidad para ciudadanos en el extranjero, personas con movilidad reducida o enfermos, además de agilizar procesos como el escrutinio.

Sin embargo, estos sistemas presentan riesgos considerables: ataques cibernéticos, errores de software, fallas de privacidad o dificultad para verificar la identidad y unicidad del votante. La votación electrónica debe enfrentar estos desafíos para garantizar confianza pública.

La incorporación de blockchain en la votación electrónica surge precisamente como respuesta a estos retos. Una cadena de bloques puede funcionar como registro público e inmutable de cada voto, ya que cada transacción (voto) queda registrada ``de manera segura, transparente y sin posibilidad de alteración''. Una vez emitido, el voto no puede ser modificado ni eliminado, garantizando la integridad del sufragio.

La descentralización elimina puntos únicos de falla: múltiples nodos validan los votos, evitando la manipulación por parte de un único administrador. Según diversas fuentes, cada voto almacenado en blockchain se cifra y se enlaza criptográficamente con los anteriores, proporcionando trazabilidad total.

Existen dos enfoques principales para implementar votación con blockchain:

\begin{itemize}
    \item Emplear criptomonedas o tokens como representación del voto.
    \item Usar blockchain únicamente como registro inmutable donde los votos digitales son almacenados.
\end{itemize}

Ambos enfoques buscan aprovechar la transparencia y descentralización del libro mayor distribuido. La literatura coincide en que blockchain mejora la confianza pública en el voto electrónico; sin embargo, también señala que no es una solución completa y requiere medidas criptográficas adicionales para preservar el anonimato (como cifrado homomórfico o zk-SNARKs), además de superar retos regulatorios e infraestructurales para su adopción masiva.

\section{Trabajos relacionados}

En la última década han surgido numerosos proyectos y pilotos de votación electrónica basados en blockchain. Uno de los casos más citados es Sierra Leona, que en 2018 empleó blockchain privada en las elecciones parlamentarias mediante la empresa Agora, registrando votos confirmados en tiempo real y reduciendo la posibilidad de manipulación.

Otro ejemplo es la ciudad de Tsukuba (Japón), que en 2018 incorporó un sistema de votación basado en blockchain para seleccionar proyectos locales. En dicha prueba, los votantes pasaban por un proceso de verificación de identidad y luego sus votos se cifraban y registraban en una red distribuida resistente a alteraciones.

De forma similar, varios estados de Estados Unidos han evaluado aplicaciones blockchain para votantes remotos, especialmente militares en el extranjero. Estonia, pionera del gobierno digital, ha investigado blockchain para reforzar su sistema de voto electrónico nacional.

La literatura académica también recoge múltiples propuestas basadas en contratos inteligentes, usualmente sobre Ethereum, que permiten automatizar procesos de votación descentralizada. También se exploran identidades digitales autosoberanas (como Sovrin o uPort) que buscan garantizar anonimato y control por parte del votante.

Sin embargo, la mayoría de estos proyectos no integran por completo la lógica de negocio de un proceso electoral real (gestión de candidatos, autorización de votantes, escrutinio). El presente trabajo busca cubrir este vacío diseñando un sistema integral alineado con los requisitos funcionales y no funcionales establecidos.

\section{Resumen del marco teórico}

En conclusión, la tecnología blockchain destaca como una herramienta idónea para reforzar sistemas de votación electrónica debido a su inmutabilidad, transparencia y consenso descentralizado. Los estudios revisados demuestran que almacenar votos en blockchain puede prevenir manipulaciones y facilitar auditorías independientes.

No obstante, se identifican requerimientos críticos que deben abordarse: asegurar el anonimato del votante, garantizar la unicidad del voto, proteger la privacidad mediante cifrado y
