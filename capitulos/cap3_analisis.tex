\chapter{Análisis del problema y requisitos}

\section{Descripción del problema}
El sistema debe permitir la realización de elecciones electrónicas asegurando que solo usuarios autorizados puedan votar, que cada usuario emita a lo sumo un voto por elección y que cada voto sea confidencial pero verificable. 

Siguiendo la lógica de negocio, únicamente los usuarios con \texttt{VoterRegister.can\_vote = true} pueden emitir un voto. Al completar el proceso, el sistema marcará \texttt{VoterRegister.has\_voted = true} para impedir votos adicionales. 

El voto es una unidad atómica e inmutable cuyo registro definitivo se publica en la blockchain. Para ello, el sistema genera un \textit{vote\_hash} criptográfico (SHA-256) que resume el contenido del voto (ID de la elección, hash del votante y lista de candidatos seleccionados). Este hash, junto con la firma digital del votante, permite validar la autenticidad del voto sin revelar su identidad real.

La base de datos relacional almacena únicamente metadatos y referencias, como: \texttt{vote\_hash}, \texttt{blockchain\_tx\_id} y \texttt{published\_at}. El problema consiste en implementar el flujo electoral completo —creación de una elección, habilitación de votantes, emisión del voto, registro en blockchain, conteo y auditoría— cumpliendo estas reglas de seguridad, integridad e inmutabilidad.

Al finalizar la elección, cualquier usuario debe poder verificar el conteo público a partir de los \textit{vote hashes} registrados en la blockchain, garantizando transparencia y confianza en los resultados.

\section{Actores y casos de uso}

Los actores principales identificados son:

\begin{itemize}
    \item \textbf{Administrador de elección:} Crea elecciones, registra candidatos y habilita votantes gestionando el \texttt{VoterRegister}.
    \item \textbf{Votante:} Usuario autenticado autorizado a votar. Solo puede emitir un voto por elección.
    \item \textbf{Candidato:} Usuario inscrito como candidato en una elección. Participa únicamente en la boleta.
    \item \textbf{Auditor o ciudadano:} Persona que verifica resultados mediante los \texttt{vote\_hash} publicados en blockchain.
\end{itemize}

El actor no se define por un rol global del usuario, sino por su relación con una elección específica.

Los principales casos de uso (CU) son:

\begin{itemize}
    \item \textbf{CU1: Autenticación.} Registro de cuenta y posterior inicio de sesión.
    \item \textbf{CU2: Crear elección (Administrador).} Define título, descripción, fechas, tipo de boleta y \texttt{max\_choices}.
    \item \textbf{CU3: Registrar candidatos (Administrador).}
    \item \textbf{CU4: Habilitar votantes (Administrador).} Asigna \texttt{can\_vote = true} y, opcionalmente, \texttt{verified\_by\_external\_system}.
    \item \textbf{CU5: Emitir voto (Votante).} Selección de opciones válidas, generación de \texttt{vote\_hash} y publicación en blockchain.
    \item \textbf{CU6: Verificación y conteo (Auditor).} Revisión y conteo mediante los hashes registrados en la blockchain.
\end{itemize}

Diagramas UML (no incluidos aquí) representarían visualmente estas interacciones.

\section{Requerimientos funcionales}

Los requerimientos funcionales (RF) del sistema son:

\begin{enumerate}
    \item \textbf{RF1. Registro y autenticación.} Permitir creación de cuentas y acceso mediante credenciales.
    \item \textbf{RF2. Gestión de elecciones.} Crear elecciones con sus atributos básicos.
    \item \textbf{RF3. Gestión de candidatos.} Registrar candidatos vinculados a una elección.
    \item \textbf{RF4. Habilitación de votantes.} Gestionar la lista de votantes autorizados mediante \texttt{VoterRegister}.
    \item \textbf{RF5. Emisión de voto.} Permitir votar una sola vez conforme a \texttt{max\_choices}.
    \item \textbf{RF6. Publicación en blockchain.} Generar hash, firmar y enviar la transacción a la blockchain.
    \item \textbf{RF7. Registro local del voto.} Guardar metadatos, \texttt{vote\_hash}, \texttt{blockchain\_tx\_id} y actualizar \texttt{has\_voted}.
    \item \textbf{RF8. Conteo de votos.} Contabilizar los votos emitidos para cada candidato.
    \item \textbf{RF9. Auditoría y verificación.} Permitir consultar y exportar \texttt{vote\_hash} para auditorías externas.
\end{enumerate}

\section{Requerimientos no funcionales}

Los requisitos no funcionales (RNF) incluyen:

\begin{itemize}
    \item \textbf{Seguridad:} Uso de HTTPS/TLS, hash seguro de contraseñas, prevención de inyección SQL y control de acceso por permisos. La blockchain garantiza inmutabilidad.
    \item \textbf{Autenticación fuerte:} Recomendación de 2FA para usuarios privilegiados.
    \item \textbf{Privacidad:} No exponer identidades reales; uso de \texttt{voter\_hash} y firmas digitales.
    \item \textbf{Escalabilidad:} Soporte para múltiples votantes concurrentes con baja latencia.
    \item \textbf{Disponibilidad:} Alta disponibilidad mediante infraestructura redundante.
    \item \textbf{Usabilidad:} Interfaz web accesible e intuitiva.
    \item \textbf{Mantenibilidad:} Código documentado, siguiendo buenas prácticas.
    \item \textbf{Legalidad:} Cumplimiento de normativas de protección de datos aplicables.
\end{itemize}

\section{Validación y restricciones}

La validación del sistema incluirá:

\begin{itemize}
    \item Verificar que un votante con \texttt{can\_vote = true} y \texttt{has\_voted = false} pueda votar correctamente.
    \item Rechazar votos duplicados o de usuarios no autorizados.
    \item Comprobar que los conteos basados en blockchain coincidan exactamente con los de la base de datos local.
\end{itemize}

Se asume que los usuarios dispondrán de navegador actual y conexión estable. No se consideran factores externos como fallos eléctricos o problemas de infraestructura fuera del alcance del sistema.

