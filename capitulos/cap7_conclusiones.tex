\chapter{Conclusiones y recomendaciones}

\section{Conclusiones generales}

Se ha desarrollado con éxito una aplicación web de votación digital que integra tecnología blockchain para asegurar la integridad electoral. El sistema cumple los objetivos planteados: ofrece autenticación de usuarios, validación de votantes autorizados (\texttt{can\_vote}), y garantiza la unicidad del voto (\texttt{has\_voted}) tal como demandan las reglas de negocio.  

Gracias al uso de la blockchain, cada voto queda registrado de forma inmutable y pública, verificable mediante su hash criptográfico (SHA-256), lo que refuerza la confianza en los resultados. Esto concuerda con lo revisado en el marco teórico, donde se destaca que la inmutabilidad y trazabilidad de la blockchain mejoran la transparencia electoral.  

En las pruebas se corroboró que, al concluir una elección, el recuento basado en la cadena fue idéntico al registrado en la base de datos interna, evidenciando la coherencia y la resistencia al fraude del sistema. En resumen, la solución propuesta demuestra cómo la blockchain puede aplicarse efectivamente en un contexto de votación electrónica, aportando las propiedades de seguridad deseadas: transparencia, inmutabilidad y auditabilidad.

\section{Limitaciones}

Entre las limitaciones del proyecto destacan su escala de prototipo y supuestos técnicos:

\begin{itemize}
    \item El sistema fue validado en un entorno controlado con una red de pruebas blockchain y un número reducido de usuarios. La performance en una red pública real con miles de votantes podría variar; por ejemplo, el tiempo de confirmación de transacciones podría aumentar en caso de congestión.
    \item No se abordó la integración con sistemas reales de identidad nacional o electoral; la habilitación de votantes es manual o simula una validación externa.
    \item No se implementaron algoritmos avanzados de preservación de privacidad (como cifrado homomórfico o pruebas de conocimiento cero), que en la literatura se mencionan como mejoras posibles.
    \item El prototipo no fue evaluado con un análisis de penetración exhaustivo en red real.
    \item Este diseño asume una infraestructura de apoyo adecuada (nodos blockchain, servidores confiables); en escenarios con limitaciones de infraestructura o conectividad, podrían surgir fallos no contemplados.
\end{itemize}

\section{Recomendaciones y trabajo futuro}

Se proponen las siguientes acciones de mejora:

\begin{itemize}
    \item \textbf{Integración con identidad digital autosoberana:} Incorporar sistemas de identidad basados en blockchain (p. ej., Sovrin o uPort) para que los votantes demuestren su derecho a voto sin revelar datos personales.
    \item \textbf{Criptografía avanzada:} Explorar esquemas de cifrado homomórfico o el uso de zk-SNARKs para asegurar que los votos puedan contarse sin exponer los datos individuales.
    \item \textbf{Escalabilidad y rendimiento:} Investigar optimizaciones de rendimiento, utilizando blockchains de capa 2 o permissionadas para reducir la latencia, y ampliar la aplicación para soportar elecciones de mayor escala.
    \item \textbf{Ampliar pruebas en escenarios reales:} Organizar pruebas piloto con usuarios reales (p. ej., elecciones estudiantiles) para evaluar usabilidad y rendimiento, así como auditorías de seguridad independientes.
    \item \textbf{Funcionalidades adicionales:} Implementar revocación de votos antes de cerrar la elección, notificaciones de estado al votante, o análisis estadístico de participación.
    \item \textbf{Alternativas tecnológicas:} Considerar otras plataformas blockchain (p. ej., Hyperledger Fabric, Algorand) que ofrezcan diferentes garantías, como menores costos de transacción o mayor privacidad.
\end{itemize}

Estas recomendaciones buscan cerrar las brechas identificadas y extender la solución hacia un producto más completo y adaptable. La investigación futura puede centrarse en integrar completamente la infraestructura de identidad y cumplir con marcos legales de votación electrónica, para eventualmente implementar un sistema de votación blockchain escalable a nivel nacional.