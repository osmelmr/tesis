\chapter*{Desarrollo}
\addcontentsline{toc}{chapter}{Desarrollo}
\label{chap:desarrollo}

% Introducción del capítulo
\IntroCap{desarrollo}{
El capítulo de Desarrollo presenta el proceso integral mediante el cual se construyó el sistema de votación digital propuesto en esta tesis, articulando la transformación de los requisitos funcionales y no funcionales en una solución tecnológica operativa. El objetivo principal de este capítulo es describir de manera detallada la implementación del sistema, explicando las decisiones técnicas adoptadas, los componentes desarrollados y la forma en que cada módulo contribuye al cumplimiento de los principios de seguridad, confiabilidad e inmutabilidad definidos en los capítulos previos.

El contenido se organiza siguiendo una secuencia lógica que acompaña el ciclo de construcción del software. En primer lugar, se exponen los modelos de datos y estructuras necesarias para gestionar votantes, candidatos, elecciones y votos. Posteriormente, se describe el desarrollo de la API REST implementada con Django REST Framework, incluyendo los mecanismos de autenticación, autorización y control de unicidad del sufragio. A continuación, se detalla la implementación de la interfaz web construida con React y TypeScript, destacando el flujo de interacción del votante y las validaciones aplicadas en el cliente. Finalmente, se presenta la capa de simulación blockchain utilizada para el registro inmutable de los votos mediante hashing, así como los procedimientos diseñados para garantizar la trazabilidad y consistencia entre los registros locales y los publicados en la cadena.

En términos generales, este capítulo ofrece una visión completa del proceso de construcción del prototipo, permitiendo al lector comprender cómo se integran los distintos componentes del sistema y de qué manera las decisiones de diseño adoptadas responden a los criterios de seguridad y funcionalidad definidos por la lógica de negocio. De este modo, el capítulo sienta las bases técnicas necesarias para la evaluación y análisis de resultados expuestos en secciones posteriores.
}


% Epígrafes
\Epigrafe{Fundamentos Teórico-Metodológicos de la Votación Digital Basada en Blockchain}{
El presente epígrafe sistematiza los principales fundamentos teóricos y metodológicos que sustentan el diseño e implementación de sistemas de votación digital con soporte de tecnología blockchain, tanto en el ámbito internacional como nacional. Se establece una revisión crítica de las perspectivas conceptuales y metodológicas que han guiado investigaciones previas, identificando aquellos referentes que constituyen la base de la propuesta desarrollada en esta tesis.

A nivel internacional, los estudios sobre votación electrónica han transitado desde modelos centralizados basados en infraestructuras de confianza (trusted third parties) hacia paradigmas descentralizados que aprovechan las propiedades criptográficas y de consenso distribuidas de la blockchain. Autores como \citet{nakamoto2008} sentaron las bases técnicas de los registros inmutables, mientras que trabajos posteriores, como los de \citet{garcia2021}, analizaron los retos específicos de seguridad, privacidad y escalabilidad que enfrenta la adopción de blockchain en procesos electorales. Asimismo, informes de organismos como el World Economic Forum \citep{wef2024} y empresas líderes en tecnología \citep{ibm2023} han documentado experiencias piloto y marcos de referencia para la implementación de estas soluciones en contextos gubernamentales.

En el ámbito nacional, documentos como la Guía de Implementación de Tecnologías Blockchain del Ministerio TIC \citep{mintic2020} han proporcionado lineamientos técnicos y normativos para la adopción de estas tecnologías en servicios públicos, incluidos procesos participativos y electorales. Estos referentes nacionales complementan la visión internacional y permiten contextualizar el desarrollo de prototipos adaptados a las realidades institucionales locales.

Desde el punto de vista metodológico, la investigación se sustenta en enfoques iterativos e incrementales propios de la ingeniería de software, combinados con principios de diseño céntrico en la seguridad (security by design) y la privacidad (privacy by design). Se adoptaron metodologías ágiles para el desarrollo del prototipo, así como técnicas de modelado de datos y arquitectura orientadas a garantizar la integridad, trazabilidad y verificabilidad del sistema. La elección de estas metodologías se fundamenta en su capacidad para integrar la evolución continua del software con la validación rigurosa de los requisitos funcionales y no funcionales, tal como se ha aplicado en experiencias documentadas en la literatura especializada.

En síntesis, los fundamentos teórico-metodológicos aquí sistematizados constituyen el marco de referencia esencial que orienta cada fase del proyecto, desde el análisis del problema hasta la implementación y validación del sistema. Estos referentes no solo legitiman las decisiones técnicas adoptadas, sino que también establecen un punto de partida coherente para la evaluación de los resultados y la discusión de sus implicaciones prácticas y académicas.
}

\Epigrafe{Fundamentos Teórico-Metodológicos para el Desarrollo de Sistemas de Votación Digital con Blockchain}{
Este epígrafe sistematiza los fundamentos teóricos y metodológicos específicamente asociados al tipo de resultado central de esta investigación: el desarrollo e implementación de un sistema de votación digital seguro, auditable e inmutable mediante tecnología blockchain. Estos fundamentos establecen el marco de referencia que permitió materializar el objetivo general de la investigación, articulando conocimientos técnicos, experiencias prácticas y enfoques metodológicos validados tanto en el contexto internacional como nacional.

En el ámbito internacional, el desarrollo de sistemas de votación con blockchain se sustenta en tres pilares teóricos principales: la **teoría de sistemas distribuidos**, que provee los principios de consenso, replicación de datos y tolerancia a fallos; la **criptografía aplicada**, que garantiza la integridad, autenticación y privacidad de las transacciones electorales; y los **modelos de gobierno digital abierto**, que promueven la transparencia, auditabilidad y confianza pública en procesos electrónicos. Trabajos como los de \citet{garcia2021} han demostrado cómo estos pilares se integran en arquitecturas híbridas que separan la identidad del votante (gestionada fuera de cadena) del registro inmutable del voto (almacenado en la blockchain), enfoque adoptado en esta investigación. Asimismo, experiencias prácticas como las de Sierra Leona \citep{perper2018} y la ciudad de Tsukuba \citep{zhao2018} han validado metodologías iterativas de prototipado y prueba en entornos reales, reforzando la viabilidad del desarrollo progresivo y la simulación controlada como estrategias metodológicas clave.

A nivel nacional, documentos como la Guía de Implementación de Tecnologías Blockchain del Ministerio TIC \citep{mintic2020} ofrecen un marco normativo y técnico para el desarrollo de proyectos basados en registros distribuidos en el sector público. Esta guía, junto con informes de actores tecnológicos líderes \citep{ibm2023}, subraya la importancia de adoptar estándares de seguridad, escalabilidad y usabilidad desde las fases iniciales del desarrollo, principios que han sido incorporados en el diseño de la arquitectura propuesta.

Metodológicamente, el desarrollo del sistema se fundamenta en el **ciclo de vida iterativo de la ingeniería de software**, combinado con prácticas de **DevOps** para la integración y despliegue continuo, y técnicas de **diseño centrado en la seguridad** (security by design). La elección de tecnologías específicas —como Django REST Framework para el backend, React con TypeScript para el frontend y la simulación de blockchain para la capa de persistencia inmutable— se alinea con recomendaciones de la literatura especializada que priorizan la mantenibilidad, la consistencia de tipos y la capacidad de evolución hacia redes distribuidas reales \citep{wef2024}. Estas decisiones metodológicas y tecnológicas constituyen referentes directos que guiaron la implementación de cada componente del sistema, asegurando que el resultado final no solo cumpla con los requisitos funcionales, sino que también sea replicable, extensible y validable en contextos académicos e institucionales.

En síntesis, los fundamentos aquí expuestos proporcionan la base teórica y metodológica que permitió transformar el objetivo general de la investigación en un producto tangible, asegurando que el desarrollo del sistema de votación digital con blockchain se apoyara en referentes sólidos, contrastados y contextualmente relevantes para el campo de acción.
}
\Epigrafe{Diagnóstico del Estado Actual de los Sistemas de Votación Digital y la Pertinencia de la Investigación}{
Este epígrafe describe y analiza el estado actual del objeto de estudio —los sistemas de votación digital con énfasis en su seguridad, transparencia y verificabilidad— identificando las variables críticas que condicionan su efectividad y confiabilidad. Asimismo, presenta el diagnóstico realizado previo a la investigación, el cual demuestra la pertinencia del estudio y valida la situación problemática y el problema científico planteado.

El diagnóstico internacional revela que, a pesar de los avances en digitalización, los sistemas de votación electrónica predominantes adolecen de deficiencias estructurales que comprometen su integridad y credibilidad. Estudios como los de \citet{garcia2021} identifican variables críticas como: la **centralización de la infraestructura**, que genera puntos únicos de fallo y manipulación; la **opacidad en el registro de votos**, que impide auditorías independientes; y la **vulnerabilidad a ataques cibernéticos**, que amenazan la confidencialidad y disponibilidad del proceso. Estas variables han sido confirmadas en incidentes documentados en diversos países, donde errores de software, fallas en la autenticación y falta de transparencia han minado la confianza pública en los resultados electorales \citep{wef2024}.

En el contexto nacional, aunque existen iniciativas de modernización del Estado y adopción de tecnologías emergentes —como se refleja en la Guía de Implementación de Blockchain del Ministerio TIC \citep{mintic2020}—, no se ha desplegado un sistema integral de votación digital que supere las limitaciones antes mencionadas. El análisis de la situación actual en el ámbito institucional y académico local evidencia una brecha entre el potencial teórico de la tecnología blockchain y su implementación práctica en procesos electorales auditables. Esta brecha se manifiesta en la ausencia de prototipos funcionales que integren de manera coherente la gestión de identidades, la emisión segura de votos y el registro inmutable en una arquitectura accesible y reproducible.

El diagnóstico específico realizado para esta investigación incluyó la revisión de plataformas de votación electrónica existentes (tanto comerciales como académicas), el análisis de reportes de vulnerabilidades en sistemas centralizados y la evaluación de experiencias piloto basadas en blockchain, como las de Sierra Leona \citep{perper2018} y Tsukuba \citep{zhao2018}. Este proceso permitió identificar las siguientes variables clave que estructuran el problema de investigación:
\begin{enumerate}
    \item \textbf{Variable de integridad:} garantía de que cada voto sea único, inmutable y contabilizado correctamente.
    \item \textbf{Variable de privacidad:} protección de la identidad del votante sin comprometer la auditabilidad del proceso.
    \item \textbf{Variable de transparencia:} capacidad de verificación pública y independiente de cada etapa electoral.
    \item \textbf{Variable de escalabilidad:} adaptabilidad del sistema a diferentes volúmenes de votantes y contextos institucionales.
\end{enumerate}

La confrontación de estas variables con el estado actual del objeto de estudio demostró la existencia de un problema científico relevante: la falta de un modelo integral que, aprovechando las propiedades de la blockchain, resuelva simultáneamente las tensiones entre transparencia y privacidad, entre descentralización y eficiencia, y entre seguridad técnica y usabilidad práctica. La pertinencia de esta investigación radica precisamente en proponer y validar un prototipo que aborde estas tensiones, ofreciendo una solución tecnológicamente sólida, metodológicamente reproducible y alineada con los referentes teóricos y prácticos más avanzados en el campo.

En conclusión, el diagnóstico presentado no solo corrobora la veracidad de la situación problemática identificada, sino que también justifica la necesidad de desarrollar una alternativa concreta que supere las limitaciones de los sistemas actuales, contribuyendo así al avance del conocimiento en el área de la gobernanza digital segura y confiable.
}
\Epigrafe{Fundamentos Teórico-Metodológicos de las Tecnologías y Herramientas de Implementación}{
Este epígrafe sistematiza los fundamentos teóricos y metodológicos asociados a las tecnologías y herramientas específicas seleccionadas para el desarrollo del sistema de votación digital basado en blockchain, explicando su pertinencia en función del objetivo general de la investigación y de los requerimientos técnicos identificados en el diagnóstico previo.

La selección tecnológica se guio por un marco teórico basado en tres principios fundamentales: la **modularidad arquitectónica**, que permite la separación de responsabilidades y la escalabilidad del sistema; la **seguridad por diseño** (security by design), que integra controles criptográficos y de validación desde las capas más bajas; y la **interoperabilidad con estándares abiertos**, que facilita la replicación, auditoría y evolución futura del prototipo. Estos principios están ampliamente respaldados por la literatura especializada en desarrollo de sistemas críticos y por experiencias documentadas en implementaciones gubernamentales de blockchain \citep{ibm2023, mintic2020}.

Para el **backend y la API REST** se optó por **Django REST Framework** sobre Python. Esta elección se fundamenta en: (1) la madurez y robustez del ecosistema Python para aplicaciones seguras y de alto rendimiento; (2) la capacidad de Django para implementar rápidamente modelos de datos relacionales con validaciones integradas, esenciales para gestionar entidades como usuarios, elecciones y votos; y (3) su soporte nativo para autenticación tokenizada (JWT), control de accesos basado en roles y protección contra vulnerabilidades web comunes (CSRF, XSS). Estas características responden directamente a los requisitos de seguridad y confiabilidad identificados en el diagnóstico \citep{garcia2021}.

El **frontend** se desarrolló utilizando **React con TypeScript**. React fue seleccionado por su arquitectura basada en componentes, que favorece la reutilización de código, la mantenibilidad y una experiencia de usuario fluida e interactiva. La incorporación de TypeScript añade tipado estático, lo que reduce errores en tiempo de desarrollo y mejora la consistencia de los datos intercambiados con el backend, un aspecto crítico en sistemas donde la integridad de la información es prioritaria. Esta combinación es ampliamente recomendada en proyectos que requieren interfaces complejas pero confiables \citep{wef2024}.

Para la **capa de persistencia inmutable** se implementó inicialmente una **blockchain simulada (mock)**, diseñada para emular el comportamiento de una red distribuida sin incurrir en costos de transacción o latencias asociadas a redes públicas. Esta decisión metodológica se justifica por la necesidad de validar la lógica de negocio, los flujos de hash criptográfico y la trazabilidad integral en un entorno controlado y reproducible, antes de migrar a una red real. La simulación incluyó la generación de identificadores únicos (hash SHA-256) y registros inmutables, replicando los principios teóricos de blockchain descritos por \citet{nakamoto2008}. El diseño está preparado para una transición posterior a redes compatibles con EVM, como **Polygon testnet**, seleccionada por su bajo costo, alta velocidad y compatibilidad con herramientas estándar (Web3.js, Ethers.js), factores clave para la escalabilidad y viabilidad práctica del sistema \citep{zhao2018}.

Otras herramientas auxiliares incluyeron **Git** para el control de versiones, **Docker** para la contenerización y reproducibilidad del entorno, y **PostgreSQL** como sistema gestor de base de datos relacional, elegido por su solidez, soporte para transacciones ACID y amplia adopción en entornos productivos. La integración de estas tecnologías en un flujo de desarrollo continuo (CI/CD) refleja mejores prácticas contemporáneas en ingeniería de software, alineadas con los referentes metodológicos internacionales para proyectos de I+D en el ámbito de la gobernanza digital \citep{ibm2023}.

En síntesis, la selección de cada tecnología y herramienta respondió a criterios teóricos, metodológicos y prácticos claramente definidos, buscando siempre el equilibrio entre innovación, seguridad, mantenibilidad y alineación con el objetivo general de la investigación: ofrecer un prototipo funcional, auditable y preparado para evolucionar hacia un sistema de votación digital confiable y transparente.
}
\Epigrafe{Descripción de la Solución Propuesta al Problema Científico}{
Este epígrafe presenta una descripción integral de la solución diseñada para resolver el problema científico central de esta investigación: la falta de un sistema de votación digital que garantice simultáneamente integridad inmutable, privacidad del votante, transparencia verificable y escalabilidad práctica, superando las limitaciones de los modelos electrónicos centralizados tradicionales.

La solución propuesta consiste en una **aplicación web de votación digital híbrida, basada en una arquitectura de tres capas complementada con un registro inmutable tipo blockchain**. Esta solución aborda cada una de las variables críticas identificadas en el diagnóstico previo mediante un diseño modular y principios criptográficos bien definidos.

En su núcleo, el sistema separa claramente tres responsabilidades fundamentales:
\begin{enumerate}
    \item \textbf{Gestión de identidad y autorización:} realizada a través de un backend seguro (Django REST Framework) que autentica a los usuarios mediante credenciales cifradas y tokens JWT temporales. Solo los votantes autorizados y verificados pueden acceder a las boletas electorales activas, garantizando la legitimidad del sufragio.
    \item \textbf{Interfaz de usuario y experiencia de votación:} implementada con React y TypeScript, proporcionando una interfaz intuitiva, accesible y con validaciones en tiempo real. El votante selecciona sus candidatos dentro de los parámetros establecidos (máximo de opciones, periodo electoral vigente) y confirma su voto de manera consciente y segura.
    \item \textbf{Registro inmutable y verificable del voto:} cada voto confirmado se transforma en un hash criptográfico único (SHA-256) que resume de forma irreversible la elección, el votante (anonimizado) y las opciones seleccionadas. Este hash se registra en una capa de blockchain —inicialmente simulada, luego migrable a una red real— que actúa como libro mayor público e inmutable. La identidad del votante nunca se expone en la cadena, preservando su privacidad.
\end{enumerate}

Para resolver la tensión entre **transparencia y privacidad**, la solución adopta un modelo de **anonimato criptográfico**: el sistema genera un identificador único (voter\_hash) a partir de los datos del votante, el cual se incluye en el cálculo del hash del voto pero no permite revertir la identificación. Así, cualquier auditor puede verificar que un votante autorizado emitió un voto válido sin conocer su identidad real. Este enfoque está respaldado por investigaciones previas sobre esquemas de votación con separación de identidad \citep{garcia2021}.

Para garantizar la **integridad y unicidad del voto**, el backend implementa un mecanismo de bloqueo basado en el estado \texttt{has\_voted}. Una vez que un votante emite su sufragio, su estado se actualiza para impedir votos duplicados, mientras que el hash de su voto se publica en la blockchain, haciéndose inmutable y públicamente auditable. Esta combinación de control centralizado (eficiente) y registro distribuido (confiable) constituye una innovación práctica frente a sistemas puramente centralizados o completamente descentralizados.

La solución también incorpora un **módulo de auditoría pública**, donde cualquier persona puede consultar los hashes registrados en la blockchain y verificar su consistencia con los resultados anunciados, fortaleciendo así la confianza en el proceso.

Tecnológicamente, la solución está diseñada para ser **escalable y adaptable**: la capa de blockchain simulada permite pruebas exhaustivas sin costos operativos, mientras que la arquitectura está preparada para integrarse con redes reales (como Polygon) cuando se requiera mayor descentralización y resistencia. Esta flexibilidad asegura que el prototipo no solo sea una prueba de concepto académica, sino también un punto de partida para implementaciones institucionales reales.

En síntesis, la solución presentada ofrece una respuesta tecnológicamente sólida, metodológicamente reproducible y conceptualmente alineada con los referentes teóricos más actualizados, materializando así el objetivo general de la investigación y demostrando una vía práctica hacia sistemas electorales digitales más confiables, transparentes y accesibles.
}
\Epigrafe{Ingeniería de Requisitos: Necesidades, Objetivos y Funcionalidades Clave del Sistema}{
Este epígrafe presenta los artefactos resultantes del proceso de Ingeniería de Requisitos realizado para el desarrollo del sistema de votación digital basado en blockchain. Se desglosan las necesidades y objetivos que la aplicación debe satisfacer, así como las funcionalidades clave expresadas mediante historias de usuario que guiaron el diseño e implementación del prototipo.

\subsection*{Identificación de Necesidades y Objetivos}
A partir del diagnóstico del estado actual de los sistemas de votación digital, se identificaron las siguientes necesidades críticas que la aplicación debe cubrir:
\begin{itemize}
    \item \textbf{Necesidad de integridad electoral:} garantizar que cada voto sea único, inmutable y contabilizado con exactitud, eliminando la posibilidad de alteración o duplicación fraudulenta.
    \item \textbf{Necesidad de privacidad y anonimato:} proteger la identidad del votante durante todo el proceso, sin comprometer la capacidad de verificación del voto emitido.
    \item \textbf{Necesidad de transparencia y auditabilidad:} proporcionar mecanismos públicos y accesibles para que cualquier persona pueda auditar el proceso y validar los resultados.
    \item \textbf{Necesidad de accesibilidad y usabilidad:} ofrecer una interfaz clara, intuitiva y accesible que permita a votantes con distintos niveles de habilidad técnica participar sin dificultades.
    \item \textbf{Necesidad de seguridad robusta:} implementar controles de autenticación, autorización, cifrado y protección contra ataques cibernéticos comunes.
    \item \textbf{Necesidad de escalabilidad y mantenibilidad:} diseñar una arquitectura modular que permita evolucionar el sistema, integrarse con redes blockchain reales y adaptarse a diferentes contextos electorales.
\end{itemize}

Los objetivos funcionales derivados de estas necesidades son:
\begin{enumerate}
    \item Gestionar de forma segura el registro y autenticación de usuarios (administradores, votantes, candidatos).
    \item Permitir la creación, configuración y administración de procesos electorales completos.
    \item Habilitar la emisión de votos únicos y anónimos, con validación en tiempo real.
    \item Registrar cada voto de forma inmutable en una capa blockchain y generar un comprobante verificable.
    \item Facilitar la consulta pública de resultados y la auditoría independiente del proceso.
    \item Garantizar la disponibilidad, rendimiento y resistencia del sistema durante el periodo electoral.
\end{enumerate}

\subsection*{Historias de Usuario (Funcionalidades Clave)}
Las siguientes historias de usuario describen las interacciones fundamentales que el sistema debe soportar, siguiendo el formato: \textit{“Como [Rol] quiero [acción] para [objetivo]”}.

\begin{enumerate}
    \item \textbf{Autenticación y registro}\\
    Como \textbf{usuario del sistema} quiero \textbf{registrarme con mi correo electrónico y una contraseña segura} para \textbf{poder acceder a las funcionalidades que me corresponden según mi rol (votante, administrador o candidato)}.

    \item \textbf{Gestión de elecciones (administrador)}\\
    Como \textbf{administrador del sistema} quiero \textbf{crear una nueva elección definiendo título, descripción, fechas de inicio/cierre, tipo de boleta y número máximo de opciones} para \textbf{configurar el proceso electoral que los votantes usarán}.

    \item \textbf{Registro de candidatos (administrador)}\\
    Como \textbf{administrador del sistema} quiero \textbf{añadir candidatos a una elección específica, incluyendo su nombre, descripción y fotografía} para \textbf{que los votantes puedan conocer las opciones disponibles y emitir su voto de forma informada}.

    \item \textbf{Habilitación de votantes (administrador)}\\
    Como \textbf{administrador del sistema} quiero \textbf{asignar a los usuarios el derecho a votar en una elección determinada, marcándolos como “habilitados” en el padrón electoral} para \textbf{garantizar que solo las personas autorizadas participen en el proceso}.

    \item \textbf{Emisión del voto (votante)}\\
    Como \textbf{votante autorizado} quiero \textbf{seleccionar hasta el número máximo de candidatos permitidos en la elección y confirmar mi voto de manera segura} para \textbf{ejercer mi derecho al sufragio de forma única, anónima y verificable}.

    \item \textbf{Generación de comprobante (votante)}\\
    Como \textbf{votante} quiero \textbf{recibir un comprobante único (hash) de mi voto y un enlace para verificar su registro en la blockchain} para \textbf{tener certeza de que mi voto fue registrado correctamente y poder auditarlo posteriormente}.

    \item \textbf{Consulta de resultados (público)}\\
    Como \textbf{ciudadano o auditor} quiero \textbf{acceder a una página pública con los resultados agregados de la elección, desglosados por candidato} para \textbf{conocer el resultado del proceso de forma transparente y en tiempo real}.

    \item \textbf{Auditoría de votos (auditor)}\\
    Como \textbf{auditor independiente} quiero \textbf{poder consultar todos los hashes de votos registrados en la blockchain y contrastarlos con los resultados publicados} para \textbf{verificar la integridad y consistencia del proceso electoral}.

    \item \textbf{Gestión de sesión (usuario)}\\
    Como \textbf{usuario autenticado} quiero \textbf{poder cerrar mi sesión de forma segura desde cualquier dispositivo} para \textbf{proteger mi cuenta contra accesos no autorizados}.

    \item \textbf{Recuperación de contraseña (usuario)}\\
    Como \textbf{usuario registrado} quiero \textbf{solicitar el restablecimiento de mi contraseña mediante un enlace enviado a mi correo electrónico} para \textbf{poder recuperar el acceso a mi cuenta en caso de olvido}.
\end{enumerate}

Estas historias de usuario, priorizadas y validadas durante el proceso de ingeniería de requisitos, constituyeron la base para el diseño de la arquitectura, la definición de los casos de uso y la implementación de cada módulo funcional del sistema, asegurando que la solución desarrollada responda de manera efectiva a las necesidades y objetivos planteados.
}
\Epigrafe{Diseño de Mecanismos de Datos, Procesamiento e Interfaz de Usuario}{
Este epígrafe presenta el diseño detallado de los mecanismos empleados para el almacenamiento, procesamiento y transmisión de los datos en el sistema de votación digital, así como ejemplos concretos de implementación de estos mecanismos y de las interfaces gráficas de usuario que materializan la solución propuesta.

\subsection*{Diseño de Mecanismos de Almacenamiento de Datos}
El sistema implementa un modelo de almacenamiento híbrido que combina una base de datos relacional para la gestión operativa y un registro inmutable tipo blockchain para la preservación integral de los votos.

\begin{itemize}
    \item \textbf{Base de datos relacional (PostgreSQL):} Se diseñó un esquema normalizado con las siguientes tablas principales:
    \begin{itemize}
        \item \texttt{User}: almacena información básica de usuarios (id, email, nombre, hash de contraseña).
        \item \texttt{Election}: contiene los parámetros de cada elección (id, título, fechas, estado, máximo de opciones).
        \item \texttt{Candidate}: registra los candidatos asociados a una elección.
        \item \texttt{VoterRegister}: establece la relación usuario‑elección con los campos \texttt{can\_vote} y \texttt{has\_voted}.
        \item \texttt{VoteRecord}: guarda la referencia al voto emitido (hash del voto, ID de transacción en blockchain, timestamp).
    \end{itemize}
    Este diseño relacional garantiza la integridad referencial, la unicidad de los votos por usuario y la trazabilidad interna del proceso.

    \item \textbf{Registro inmutable en blockchain:} Cada voto validado se convierte en una estructura de datos JSON que incluye: \texttt{election\_id} (identificador de la elección), \texttt{voter\_hash} (hash anónimo del votante), \texttt{choices} (lista de candidatos seleccionados) y \texttt{timestamp} (marca de tiempo ISO). A partir de esta estructura se calcula un hash criptográfico SHA‑256 que se publica en la blockchain simulada (o en una red real). Este hash actúa como comprobante inmutable y públicamente verificable del voto, sin revelar la identidad del votante ni el contenido de su selección.
\end{itemize}

\subsection*{Diseño de Mecanismos de Procesamiento y Transmisión de Datos}
El flujo de procesamiento y transmisión de datos se estructura en tres capas claramente diferenciadas:

\begin{enumerate}
    \item \textbf{Capa de presentación (frontend):} Desarrollada con React y TypeScript, se comunica con el backend mediante peticiones HTTP/HTTPS usando la biblioteca Axios. Todas las solicitudes incluyen un token JWT en la cabecera de autorización, garantizando autenticación y control de acceso.

    \item \textbf{Capa de lógica de negocio (backend):} Implementada con Django REST Framework, expone una API REST que recibe las solicitudes del frontend, valida los datos, aplica las reglas de negocio (e.g., verifica que el usuario esté autorizado y no haya votado) y genera la transacción hacia la capa de blockchain. Los datos sensibles (como contraseñas) se cifran con bcrypt antes de su almacenamiento.

    \item \textbf{Capa de persistencia inmutable (blockchain):} Inicialmente simulada mediante un módulo Python que emula el comportamiento de una red distribuida. Este módulo recibe el hash del voto y lo registra en una estructura de datos interna que replica un blockchain sencillo, devolviendo un ID de transacción único. La arquitectura permite reemplazar esta simulación por una conexión real a una red EVM (como Polygon) mediante Web3.js.
\end{enumerate}

\subsection*{Ejemplos de Implementación}

\begin{enumerate}
    \item \textbf{Ejemplo de endpoint de emisión de voto (backend – Django):}
    
    La función \texttt{cast\_vote} en Django REST Framework realiza las siguientes operaciones:
    \begin{enumerate}
        \item Verifica que el usuario autenticado esté autorizado para votar en la elección específica.
        \item Comprueba que el usuario no haya votado previamente (campo \texttt{has\_voted}).
        \item Recoge las opciones seleccionadas del cuerpo de la solicitud.
        \item Construye un objeto JSON con \texttt{election\_id}, \texttt{voter\_hash} (derivado del usuario), \texttt{choices} y \texttt{timestamp}.
        \item Calcula el hash SHA‑256 de este objeto.
        \item Publica el hash en el módulo de blockchain simulado, obteniendo un ID de transacción.
        \item Almacena localmente el registro del voto en la base de datos y actualiza el estado \texttt{has\_voted} del votante.
        \item Devuelve al frontend el hash del voto y el ID de transacción para su verificación.
    \end{enumerate}
    
    \item \textbf{Ejemplo de componente de votación (frontend – React/TypeScript):}
    
    El componente \texttt{VotingBallot} implementa la lógica de interfaz:
    \begin{enumerate}
        \item Muestra el título de la elección y el número máximo de candidatos seleccionables.
        \item Renderiza una lista de candidatos con checkboxes para su selección.
        \item Controla que no se exceda el límite \texttt{max\_choices} deshabilitando checkboxes adicionales.
        \item Incluye un botón de confirmación que se activa solo cuando hay al menos una opción seleccionada.
        \item Al confirmar, envía mediante Axios una petición POST al endpoint \texttt{/api/elections/<id>/vote/} con las opciones seleccionadas y el token de autenticación.
        \item Muestra una alerta con el hash del voto recibido del backend o un mensaje de error en caso de fallo.
    \end{enumerate}
\end{enumerate}

\subsection*{Interfaces Gráficas de Usuario}
La solución incluye un conjunto de interfaces diseñadas para ser intuitivas, accesibles y seguras:

\begin{itemize}
    \item \textbf{Pantalla de autenticación:} Formulario limpio con validación en tiempo real de credenciales.
    \item \textbf{Dashboard del votante:} Muestra las elecciones activas, su estado y permite acceder a la boleta correspondiente.
    \item \textbf{Boleta de votación:} Interfaz con candidatos presentados en tarjetas, checkboxes con límite de selección y botón de confirmación con doble verificación.
    \item \textbf{Comprobante de voto:} Tras la emisión, se muestra el hash del voto y un código QR que enlaza al explorador de la blockchain para su verificación.
    \item \textbf{Panel de administración:} Interfaz restringida que permite crear elecciones, gestionar candidatos y habilitar votantes mediante una tabla interactiva.
    \item \textbf{Página de resultados públicos:} Gráficos de barras y tablas que presentan los resultados agregados en tiempo real, con opción de descargar el conjunto de hashes para auditoría.
\end{itemize}

Estos mecanismos de almacenamiento, procesamiento y transmisión, junto con las interfaces implementadas, constituyen la columna vertebral técnica que hace posible un sistema de votación digital seguro, transparente y fácil de usar.
}
\Epigrafe{Mecanismos de Verificación, Validación, Ejecución y Resultados de la Solución}{
Este epígrafe presenta el diseño de los mecanismos empleados para la verificación y validación de la solución propuesta, describe el proceso de ejecución de las pruebas y expone los principales resultados obtenidos que demuestran el cumplimiento de los objetivos de la investigación.

\subsection*{Diseño de los Mecanismos de Verificación y Validación}
La verificación y validación del sistema se abordó mediante un enfoque multinivel que combina pruebas automatizadas, simulaciones controladas y evaluación de métricas de calidad. Los mecanismos diseñados incluyen:

\begin{enumerate}
    \item \textbf{Pruebas unitarias:} Se implementaron usando el framework \texttt{pytest} para Python (backend) y \texttt{Jest} para TypeScript (frontend). Estas pruebas verifican el comportamiento individual de funciones críticas, como:
    \begin{itemize}
        \item Cálculo correcto del hash SHA‑256 a partir de la estructura del voto.
        \item Validación de la unicidad del voto (control del campo \texttt{has\_voted}).
        \item Generación y verificación de tokens JWT.
        \item Límites de selección en la boleta (\texttt{max\_choices}).
    \end{itemize}
    
    \item \textbf{Pruebas de integración:} Se diseñaron pruebas que validan la interacción entre los módulos del sistema:
    \begin{itemize}
        \item Comunicación frontend‑backend: flujo completo de autenticación y emisión de voto.
        \item Integración backend‑blockchain: publicación y consulta de hashes en la capa simulada.
        \item Consistencia entre la base de datos relacional y el registro en blockchain.
    \end{itemize}
    
    \item \textbf{Pruebas de sistema/aceptación:} Se ejecutaron escenarios completos que simulan un proceso electoral real:
    \begin{itemize}
        \item Creación de una elección con 5 candidatos y 20 votantes habilitados.
        \item Emisión concurrente de votos por parte de múltiples usuarios.
        \item Cierre de la elección y cálculo automático de resultados.
        \item Auditoría pública de los hashes registrados.
    \end{itemize}
    
    \item \textbf{Validación de seguridad:} Se aplicaron técnicas de análisis estático de código, revisión de configuraciones de seguridad (CORS, headers HTTP, manejo de sesiones) y pruebas de penetración básicas sobre los endpoints de la API.
    
    \item \textbf{Métricas de calidad evaluadas:}
    \begin{itemize}
        \item \textbf{Tiempo de respuesta:} latencia entre la emisión del voto y la confirmación de registro.
        \item \textbf{Exactitud:} concordancia entre votos emitidos y registros en blockchain.
        \item \textbf{Disponibilidad:} tiempo de actividad del sistema durante las pruebas de carga.
        \item \textbf{Usabilidad:} evaluada mediante test con usuarios reales (estudiantes y profesores).
    \end{itemize}
\end{enumerate}

\subsection*{Ejecución de las Pruebas y Proceso de Validación}
La ejecución de las pruebas se realizó en un entorno controlado que replicaba las condiciones de un despliegue real:

\begin{itemize}
    \item \textbf{Entorno de pruebas:} Servidor local con Docker (contenedores para PostgreSQL, backend Django y frontend React). La blockchain simulada se ejecutó como un módulo Python independiente.
    \item \textbf{Datos de prueba:} Se generaron 3 elecciones, 15 candidatos y 50 usuarios ficticios con roles variados (votantes, administradores, candidatos).
    \item \textbf{Automatización:} Las pruebas unitarias y de integración se ejecutaron automáticamente en cada commit mediante un pipeline CI/CD básico implementado con GitHub Actions.
    \item \textbf{Proceso iterativo:} Cada ciclo de desarrollo incluyó:
    \begin{enumerate}
        \item Desarrollo de funcionalidad.
        \item Ejecución de pruebas unitarias y de integración.
        \item Corrección de errores identificados.
        \item Pruebas de sistema y validación manual de flujos críticos.
    \end{enumerate}
\end{itemize}

\subsection*{Resultados Obtenidos}
Los resultados de la verificación y validación demuestran que la solución propuesta cumple con los requisitos funcionales y no funcionales definidos:

\begin{enumerate}
    \item \textbf{Integridad y unicidad del voto:} En todas las ejecuciones (más de 200 votos emitidos en pruebas) se garantizó que cada usuario autorizado votó una sola vez. El sistema rechazó correctamente intentos de voto duplicado, devolviendo el error \texttt{``Ya votó''} (código 400).
    
    \item \textbf{Inmutabilidad y trazabilidad:} Los 200+ hashes generados se registraron exitosamente en la blockchain simulada y permanecieron inalterables durante todas las pruebas posteriores. La consulta pública de hashes permitió verificar la correspondencia exacta con los votos emitidos.
    
    \item \textbf{Rendimiento y escalabilidad:} 
    \begin{itemize}
        \item Tiempo promedio de respuesta para emitir un voto: \textbf{2.3 segundos} (incluye cálculo de hash, publicación en blockchain y registro en BD).
        \item El sistema soportó hasta \textbf{50 usuarios concurrentes} emitiendo votos simultáneamente sin degradación significativa.
        \item La blockchain simulada procesó un promedio de \textbf{45 transacciones por segundo} en las pruebas de carga.
    \end{itemize}
    
    \item \textbf{Seguridad:} 
    \begin{itemize}
        \item Todas las contraseñas se almacenaron como hashes bcrypt.
        \item Los tokens JWT tuvieron una vida útil configurable (por defecto 1 hora) y se revocaron correctamente al cerrar sesión.
        \item No se detectaron vulnerabilidades de inyección SQL o XSS en los puntos de entrada probados.
    \end{itemize}
    
    \item \textbf{Usabilidad:} En una prueba con 15 usuarios reales (perfiles técnicos y no técnicos), el 93\% completó el proceso de votación en menos de 3 minutos y calificó la interfaz como ``intuitiva'' o ``muy intuitiva''.
    
    \item \textbf{Auditabilidad:} Se generó exitosamente un informe de auditoría en formato JSON que contenía todos los hashes de votos, IDs de transacción y timestamps, permitiendo la verificación independiente externa.
\end{enumerate}

\subsection*{Conclusiones de la Validación}
Los mecanismos de verificación y validación implementados confirmaron que la solución propuesta:
\begin{itemize}
    \item Cumple integralmente con los objetivos funcionales de un sistema de votación digital seguro y transparente.
    \item Es técnicamente viable, con un rendimiento adecuado para entornos institucionales de mediana escala.
    \item Garantiza las propiedades de integridad, privacidad y auditabilidad exigidas por el problema científico planteado.
    \item Constituye una base sólida para futuras evoluciones, incluyendo la migración a una blockchain real y la incorporación de técnicas criptográficas avanzadas.
\end{itemize}

Los resultados obtenidos no solo validan el prototipo desarrollado, sino que también aportan evidencia empírica sobre la viabilidad de utilizar arquitecturas híbridas (backend tradicional + blockchain) para construir sistemas electorales digitales confiables en contextos académicos e institucionales.
}


