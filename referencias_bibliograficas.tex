\BibliRefer{
\vspace{0.5cm}
\begin{itemize}

  \item \textbf{Artículo de revista electrónica (sin DOI):}\\
  García-Font, V. \& Rifa-Pous, H. (2021). \textit{Uso y retos de blockchain en plataformas de votación electrónica}. Seguridad Informática, 6, 255--262. Recuperado de https://openaccess.uoc.edu/server/api/core/bitstreams/946c87bf-fd8a-4e11-a4ff-9b81bbdd6fba/content

  \item \textbf{Artículo técnico / white paper institucional:}\\
  IBM Blockchain Team. (2023). \textit{Blockchain for Government and Public Services: Secure Digital Governance}. IBM.

  \item \textbf{Informe gubernamental:}\\
  Ministerio TIC. (2020). \textit{Guía de Implementación de Tecnologías Blockchain en Servicios Públicos}. Gobierno de Colombia.

  \item \textbf{Artículo académico / reporte seminal:}\\
  Nakamoto, S. (2008). \textit{Bitcoin: A Peer-to-Peer Electronic Cash System}. Recuperado de https://bitcoin.org/bitcoin.pdf

  \item \textbf{Artículo de periódico:}\\
  Perper, R. (2018). Sierra Leone just used blockchain to tally election votes. \textit{Business Insider}. Recuperado de https://www.businessinsider.com/sierra-leone-blockchain-election-vote-tally-2018

  \item \textbf{Reporte técnico independiente:}\\
  Unknown Gravity Research Group. (2022). \textit{Cybersecurity, Decentralization and Trust Models for Distributed Voting}. Technical Report.

  \item \textbf{Informe internacional (organización global):}\\
  World Economic Forum. (2024). \textit{Decentralized Systems for Elections: Opportunities and Risks}. Global Future Council on Digital Governance.

  \item \textbf{Artículo de reportaje tecnológico:}\\
  Zhao, W. (2018). Tsukuba becomes first Japanese city to test blockchain in voting system. \textit{CoinDesk}. Recuperado de https://www.coindesk.com/tsukuba-japan-blockchain-voting

\end{itemize}

}
