\conclusions{
A partir del desarrollo y validación del sistema de votación digital basado en blockchain, se derivan las siguientes conclusiones fundamentales:

\textbf{1. Sistematización del estado del arte referido al objeto de estudio y el campo de acción}
La revisión teórica realizada confirmó que la tecnología blockchain constituye un referente sólido para abordar los desafíos de integridad y transparencia en sistemas de votación electrónica. Los estudios analizados, como los de García-Font y Rifa-Pous (2021), demuestran que las propiedades de inmutabilidad, descentralización y trazabilidad de la blockchain pueden mitigar vulnerabilidades inherentes a los modelos centralizados tradicionales. Asimismo, experiencias prácticas internacionales, como las implementadas en Sierra Leona (Perper, 2018) y Tsukuba (Zhao, 2018), validan la viabilidad técnica de estas soluciones en contextos reales. En el ámbito nacional, la Guía de Implementación de Blockchain del Ministerio TIC (2020) establece un marco normativo y técnico que respalda la adopción progresiva de estas tecnologías en servicios públicos electorales.

\textbf{2. Diagnóstico del estado actual del objeto de estudio}
El diagnóstico evidenció que los sistemas de votación electrónica predominantes presentan deficiencias críticas en tres dimensiones: (a) integridad, debido a la posibilidad de manipulación en servidores centralizados; (b) transparencia, por la ausencia de mecanismos de auditoría pública independiente; y (c) privacidad, al existir riesgos de correlación entre identidad del votante y su sufragio. Estas limitaciones justificaron la necesidad de desarrollar una solución alternativa que, aprovechando las ventajas de la blockchain, superara dichas brechas sin comprometer la usabilidad ni la escalabilidad del proceso.

\textbf{3. Principales aspectos del análisis, diseño e implementación de la solución}
El análisis de requisitos permitió definir una arquitectura híbrida que separa claramente la gestión de identidad (backend seguro) del registro inmutable del voto (blockchain). El diseño implementado integra tres componentes principales: (a) una API REST desarrollada con Django que gestiona autenticación, autorización y lógica electoral; (b) una interfaz web en React/TypeScript que ofrece una experiencia de usuario intuitiva y validaciones en tiempo real; y (c) una capa de blockchain simulada que registra hashes criptográficos únicos e inalterables por cada voto. Esta arquitectura garantiza simultáneamente anonimato del votante (mediante voter\_hash) y verificabilidad pública (mediante consulta de hashes).

\textbf{4. Principales resultados de la validación de la solución propuesta}
Las pruebas exhaustivas realizadas demostraron que el sistema cumple satisfactoriamente con los objetivos planteados: (a) \textbf{Integridad y unicidad:} se procesaron más de 200 votos sin duplicaciones ni alteraciones, con rechazo automático de intentos de voto múltiple; (b) \textbf{Inmutabilidad y trazabilidad:} todos los hashes generados se registraron exitosamente en la blockchain simulada y permanecieron accesibles para auditoría; (c) \textbf{Rendimiento:} el tiempo promedio de respuesta fue de 2.3 segundos por voto, con soporte para 50 usuarios concurrentes; (d) \textbf{Usabilidad:} el 93\% de los usuarios de prueba calificaron la interfaz como intuitiva y completaron el proceso en menos de 3 minutos. Estos resultados confirman que el prototipo desarrollado no solo es técnicamente viable, sino que también ofrece garantías superiores de transparencia y seguridad en comparación con soluciones electrónicas tradicionales.

En síntesis, esta investigación aporta evidencia concreta sobre la factibilidad de implementar sistemas de votación digital basados en blockchain para contextos académicos e institucionales, estableciendo un referente metodológico y técnico replicable para futuros desarrollos en el campo de la gobernanza digital transparente.
}