\Epigrafe{Diseño de Mecanismos de Datos, Procesamiento e Interfaz de Usuario}{
Este epígrafe presenta el diseño detallado de los mecanismos empleados para el almacenamiento, procesamiento y transmisión de los datos en el sistema de votación digital, así como ejemplos concretos de implementación de estos mecanismos y de las interfaces gráficas de usuario que materializan la solución propuesta.

\subsection*{Diseño de Mecanismos de Almacenamiento de Datos}
El sistema implementa un modelo de almacenamiento híbrido que combina una base de datos relacional para la gestión operativa y un registro inmutable tipo blockchain para la preservación integral de los votos.

\begin{itemize}
    \item \textbf{Base de datos relacional (PostgreSQL):} Se diseñó un esquema normalizado con las siguientes tablas principales:
    \begin{itemize}
        \item \texttt{User}: almacena información básica de usuarios (id, email, nombre, hash de contraseña).
        \item \texttt{Election}: contiene los parámetros de cada elección (id, título, fechas, estado, máximo de opciones).
        \item \texttt{Candidate}: registra los candidatos asociados a una elección.
        \item \texttt{VoterRegister}: establece la relación usuario‑elección con los campos \texttt{can\_vote} y \texttt{has\_voted}.
        \item \texttt{VoteRecord}: guarda la referencia al voto emitido (hash del voto, ID de transacción en blockchain, timestamp).
    \end{itemize}
    Este diseño relacional garantiza la integridad referencial, la unicidad de los votos por usuario y la trazabilidad interna del proceso.

    \item \textbf{Registro inmutable en blockchain:} Cada voto validado se convierte en una estructura de datos JSON que incluye: \texttt{election\_id} (identificador de la elección), \texttt{voter\_hash} (hash anónimo del votante), \texttt{choices} (lista de candidatos seleccionados) y \texttt{timestamp} (marca de tiempo ISO). A partir de esta estructura se calcula un hash criptográfico SHA‑256 que se publica en la blockchain simulada (o en una red real). Este hash actúa como comprobante inmutable y públicamente verificable del voto, sin revelar la identidad del votante ni el contenido de su selección.
\end{itemize}

\subsection*{Diseño de Mecanismos de Procesamiento y Transmisión de Datos}
El flujo de procesamiento y transmisión de datos se estructura en tres capas claramente diferenciadas:

\begin{enumerate}
    \item \textbf{Capa de presentación (frontend):} Desarrollada con React y TypeScript, se comunica con el backend mediante peticiones HTTP/HTTPS usando la biblioteca Axios. Todas las solicitudes incluyen un token JWT en la cabecera de autorización, garantizando autenticación y control de acceso.

    \item \textbf{Capa de lógica de negocio (backend):} Implementada con Django REST Framework, expone una API REST que recibe las solicitudes del frontend, valida los datos, aplica las reglas de negocio (e.g., verifica que el usuario esté autorizado y no haya votado) y genera la transacción hacia la capa de blockchain. Los datos sensibles (como contraseñas) se cifran con bcrypt antes de su almacenamiento.

    \item \textbf{Capa de persistencia inmutable (blockchain):} Inicialmente simulada mediante un módulo Python que emula el comportamiento de una red distribuida. Este módulo recibe el hash del voto y lo registra en una estructura de datos interna que replica un blockchain sencillo, devolviendo un ID de transacción único. La arquitectura permite reemplazar esta simulación por una conexión real a una red EVM (como Polygon) mediante Web3.js.
\end{enumerate}

\subsection*{Ejemplos de Implementación}

\begin{enumerate}
    \item \textbf{Ejemplo de endpoint de emisión de voto (backend – Django):}
    
    La función \texttt{cast\_vote} en Django REST Framework realiza las siguientes operaciones:
    \begin{enumerate}
        \item Verifica que el usuario autenticado esté autorizado para votar en la elección específica.
        \item Comprueba que el usuario no haya votado previamente (campo \texttt{has\_voted}).
        \item Recoge las opciones seleccionadas del cuerpo de la solicitud.
        \item Construye un objeto JSON con \texttt{election\_id}, \texttt{voter\_hash} (derivado del usuario), \texttt{choices} y \texttt{timestamp}.
        \item Calcula el hash SHA‑256 de este objeto.
        \item Publica el hash en el módulo de blockchain simulado, obteniendo un ID de transacción.
        \item Almacena localmente el registro del voto en la base de datos y actualiza el estado \texttt{has\_voted} del votante.
        \item Devuelve al frontend el hash del voto y el ID de transacción para su verificación.
    \end{enumerate}
    
    \item \textbf{Ejemplo de componente de votación (frontend – React/TypeScript):}
    
    El componente \texttt{VotingBallot} implementa la lógica de interfaz:
    \begin{enumerate}
        \item Muestra el título de la elección y el número máximo de candidatos seleccionables.
        \item Renderiza una lista de candidatos con checkboxes para su selección.
        \item Controla que no se exceda el límite \texttt{max\_choices} deshabilitando checkboxes adicionales.
        \item Incluye un botón de confirmación que se activa solo cuando hay al menos una opción seleccionada.
        \item Al confirmar, envía mediante Axios una petición POST al endpoint \texttt{/api/elections/<id>/vote/} con las opciones seleccionadas y el token de autenticación.
        \item Muestra una alerta con el hash del voto recibido del backend o un mensaje de error en caso de fallo.
    \end{enumerate}
\end{enumerate}

\subsection*{Interfaces Gráficas de Usuario}
La solución incluye un conjunto de interfaces diseñadas para ser intuitivas, accesibles y seguras:

\begin{itemize}
    \item \textbf{Pantalla de autenticación:} Formulario limpio con validación en tiempo real de credenciales.
    \item \textbf{Dashboard del votante:} Muestra las elecciones activas, su estado y permite acceder a la boleta correspondiente.
    \item \textbf{Boleta de votación:} Interfaz con candidatos presentados en tarjetas, checkboxes con límite de selección y botón de confirmación con doble verificación.
    \item \textbf{Comprobante de voto:} Tras la emisión, se muestra el hash del voto y un código QR que enlaza al explorador de la blockchain para su verificación.
    \item \textbf{Panel de administración:} Interfaz restringida que permite crear elecciones, gestionar candidatos y habilitar votantes mediante una tabla interactiva.
    \item \textbf{Página de resultados públicos:} Gráficos de barras y tablas que presentan los resultados agregados en tiempo real, con opción de descargar el conjunto de hashes para auditoría.
\end{itemize}

Estos mecanismos de almacenamiento, procesamiento y transmisión, junto con las interfaces implementadas, constituyen la columna vertebral técnica que hace posible un sistema de votación digital seguro, transparente y fácil de usar.
}