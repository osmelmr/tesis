\Epigrafe{Fundamentos Teórico-Metodológicos de las Tecnologías y Herramientas de Implementación}{
Este epígrafe sistematiza los fundamentos teóricos y metodológicos asociados a las tecnologías y herramientas específicas seleccionadas para el desarrollo del sistema de votación digital basado en blockchain, explicando su pertinencia en función del objetivo general de la investigación y de los requerimientos técnicos identificados en el diagnóstico previo.

La selección tecnológica se guio por un marco teórico basado en tres principios fundamentales: la **modularidad arquitectónica**, que permite la separación de responsabilidades y la escalabilidad del sistema; la **seguridad por diseño** (security by design), que integra controles criptográficos y de validación desde las capas más bajas; y la **interoperabilidad con estándares abiertos**, que facilita la replicación, auditoría y evolución futura del prototipo. Estos principios están ampliamente respaldados por la literatura especializada en desarrollo de sistemas críticos y por experiencias documentadas en implementaciones gubernamentales de blockchain \citep{ibm2023, mintic2020}.

Para el **backend y la API REST** se optó por **Django REST Framework** sobre Python. Esta elección se fundamenta en: (1) la madurez y robustez del ecosistema Python para aplicaciones seguras y de alto rendimiento; (2) la capacidad de Django para implementar rápidamente modelos de datos relacionales con validaciones integradas, esenciales para gestionar entidades como usuarios, elecciones y votos; y (3) su soporte nativo para autenticación tokenizada (JWT), control de accesos basado en roles y protección contra vulnerabilidades web comunes (CSRF, XSS). Estas características responden directamente a los requisitos de seguridad y confiabilidad identificados en el diagnóstico \citep{garcia2021}.

El **frontend** se desarrolló utilizando **React con TypeScript**. React fue seleccionado por su arquitectura basada en componentes, que favorece la reutilización de código, la mantenibilidad y una experiencia de usuario fluida e interactiva. La incorporación de TypeScript añade tipado estático, lo que reduce errores en tiempo de desarrollo y mejora la consistencia de los datos intercambiados con el backend, un aspecto crítico en sistemas donde la integridad de la información es prioritaria. Esta combinación es ampliamente recomendada en proyectos que requieren interfaces complejas pero confiables \citep{wef2024}.

Para la **capa de persistencia inmutable** se implementó inicialmente una **blockchain simulada (mock)**, diseñada para emular el comportamiento de una red distribuida sin incurrir en costos de transacción o latencias asociadas a redes públicas. Esta decisión metodológica se justifica por la necesidad de validar la lógica de negocio, los flujos de hash criptográfico y la trazabilidad integral en un entorno controlado y reproducible, antes de migrar a una red real. La simulación incluyó la generación de identificadores únicos (hash SHA-256) y registros inmutables, replicando los principios teóricos de blockchain descritos por \citet{nakamoto2008}. El diseño está preparado para una transición posterior a redes compatibles con EVM, como **Polygon testnet**, seleccionada por su bajo costo, alta velocidad y compatibilidad con herramientas estándar (Web3.js, Ethers.js), factores clave para la escalabilidad y viabilidad práctica del sistema \citep{zhao2018}.

Otras herramientas auxiliares incluyeron **Git** para el control de versiones, **Docker** para la contenerización y reproducibilidad del entorno, y **PostgreSQL** como sistema gestor de base de datos relacional, elegido por su solidez, soporte para transacciones ACID y amplia adopción en entornos productivos. La integración de estas tecnologías en un flujo de desarrollo continuo (CI/CD) refleja mejores prácticas contemporáneas en ingeniería de software, alineadas con los referentes metodológicos internacionales para proyectos de I+D en el ámbito de la gobernanza digital \citep{ibm2023}.

En síntesis, la selección de cada tecnología y herramienta respondió a criterios teóricos, metodológicos y prácticos claramente definidos, buscando siempre el equilibrio entre innovación, seguridad, mantenibilidad y alineación con el objetivo general de la investigación: ofrecer un prototipo funcional, auditable y preparado para evolucionar hacia un sistema de votación digital confiable y transparente.
}