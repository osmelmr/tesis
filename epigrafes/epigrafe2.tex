\Epigrafe{Fundamentos Teórico-Metodológicos para el Desarrollo de Sistemas de Votación Digital con Blockchain}{
Este epígrafe sistematiza los fundamentos teóricos y metodológicos específicamente asociados al tipo de resultado central de esta investigación: el desarrollo e implementación de un sistema de votación digital seguro, auditable e inmutable mediante tecnología blockchain. Estos fundamentos establecen el marco de referencia que permitió materializar el objetivo general de la investigación, articulando conocimientos técnicos, experiencias prácticas y enfoques metodológicos validados tanto en el contexto internacional como nacional.

En el ámbito internacional, el desarrollo de sistemas de votación con blockchain se sustenta en tres pilares teóricos principales: la **teoría de sistemas distribuidos**, que provee los principios de consenso, replicación de datos y tolerancia a fallos; la **criptografía aplicada**, que garantiza la integridad, autenticación y privacidad de las transacciones electorales; y los **modelos de gobierno digital abierto**, que promueven la transparencia, auditabilidad y confianza pública en procesos electrónicos. Trabajos como los de \citet{garcia2021} han demostrado cómo estos pilares se integran en arquitecturas híbridas que separan la identidad del votante (gestionada fuera de cadena) del registro inmutable del voto (almacenado en la blockchain), enfoque adoptado en esta investigación. Asimismo, experiencias prácticas como las de Sierra Leona \citep{perper2018} y la ciudad de Tsukuba \citep{zhao2018} han validado metodologías iterativas de prototipado y prueba en entornos reales, reforzando la viabilidad del desarrollo progresivo y la simulación controlada como estrategias metodológicas clave.

A nivel nacional, documentos como la Guía de Implementación de Tecnologías Blockchain del Ministerio TIC \citep{mintic2020} ofrecen un marco normativo y técnico para el desarrollo de proyectos basados en registros distribuidos en el sector público. Esta guía, junto con informes de actores tecnológicos líderes \citep{ibm2023}, subraya la importancia de adoptar estándares de seguridad, escalabilidad y usabilidad desde las fases iniciales del desarrollo, principios que han sido incorporados en el diseño de la arquitectura propuesta.

Metodológicamente, el desarrollo del sistema se fundamenta en el **ciclo de vida iterativo de la ingeniería de software**, combinado con prácticas de **DevOps** para la integración y despliegue continuo, y técnicas de **diseño centrado en la seguridad** (security by design). La elección de tecnologías específicas —como Django REST Framework para el backend, React con TypeScript para el frontend y la simulación de blockchain para la capa de persistencia inmutable— se alinea con recomendaciones de la literatura especializada que priorizan la mantenibilidad, la consistencia de tipos y la capacidad de evolución hacia redes distribuidas reales \citep{wef2024}. Estas decisiones metodológicas y tecnológicas constituyen referentes directos que guiaron la implementación de cada componente del sistema, asegurando que el resultado final no solo cumpla con los requisitos funcionales, sino que también sea replicable, extensible y validable en contextos académicos e institucionales.

En síntesis, los fundamentos aquí expuestos proporcionan la base teórica y metodológica que permitió transformar el objetivo general de la investigación en un producto tangible, asegurando que el desarrollo del sistema de votación digital con blockchain se apoyara en referentes sólidos, contrastados y contextualmente relevantes para el campo de acción.
}