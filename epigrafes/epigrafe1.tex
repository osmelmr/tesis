\Epigrafe{Fundamentos Teórico-Metodológicos de la Votación Digital Basada en Blockchain}{
El presente epígrafe sistematiza los principales fundamentos teóricos y metodológicos que sustentan el diseño e implementación de sistemas de votación digital con soporte de tecnología blockchain, tanto en el ámbito internacional como nacional. Se establece una revisión crítica de las perspectivas conceptuales y metodológicas que han guiado investigaciones previas, identificando aquellos referentes que constituyen la base de la propuesta desarrollada en esta tesis.

A nivel internacional, los estudios sobre votación electrónica han transitado desde modelos centralizados basados en infraestructuras de confianza (trusted third parties) hacia paradigmas descentralizados que aprovechan las propiedades criptográficas y de consenso distribuidas de la blockchain. Autores como \citet{nakamoto2008} sentaron las bases técnicas de los registros inmutables, mientras que trabajos posteriores, como los de \citet{garcia2021}, analizaron los retos específicos de seguridad, privacidad y escalabilidad que enfrenta la adopción de blockchain en procesos electorales. Asimismo, informes de organismos como el World Economic Forum \citep{wef2024} y empresas líderes en tecnología \citep{ibm2023} han documentado experiencias piloto y marcos de referencia para la implementación de estas soluciones en contextos gubernamentales.

En el ámbito nacional, documentos como la Guía de Implementación de Tecnologías Blockchain del Ministerio TIC \citep{mintic2020} han proporcionado lineamientos técnicos y normativos para la adopción de estas tecnologías en servicios públicos, incluidos procesos participativos y electorales. Estos referentes nacionales complementan la visión internacional y permiten contextualizar el desarrollo de prototipos adaptados a las realidades institucionales locales.

Desde el punto de vista metodológico, la investigación se sustenta en enfoques iterativos e incrementales propios de la ingeniería de software, combinados con principios de diseño céntrico en la seguridad (security by design) y la privacidad (privacy by design). Se adoptaron metodologías ágiles para el desarrollo del prototipo, así como técnicas de modelado de datos y arquitectura orientadas a garantizar la integridad, trazabilidad y verificabilidad del sistema. La elección de estas metodologías se fundamenta en su capacidad para integrar la evolución continua del software con la validación rigurosa de los requisitos funcionales y no funcionales, tal como se ha aplicado en experiencias documentadas en la literatura especializada.

En síntesis, los fundamentos teórico-metodológicos aquí sistematizados constituyen el marco de referencia esencial que orienta cada fase del proyecto, desde el análisis del problema hasta la implementación y validación del sistema. Estos referentes no solo legitiman las decisiones técnicas adoptadas, sino que también establecen un punto de partida coherente para la evaluación de los resultados y la discusión de sus implicaciones prácticas y académicas.
}
