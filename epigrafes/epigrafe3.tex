\Epigrafe{Diagnóstico del Estado Actual de los Sistemas de Votación Digital y la Pertinencia de la Investigación}{
Este epígrafe describe y analiza el estado actual del objeto de estudio —los sistemas de votación digital con énfasis en su seguridad, transparencia y verificabilidad— identificando las variables críticas que condicionan su efectividad y confiabilidad. Asimismo, presenta el diagnóstico realizado previo a la investigación, el cual demuestra la pertinencia del estudio y valida la situación problemática y el problema científico planteado.

El diagnóstico internacional revela que, a pesar de los avances en digitalización, los sistemas de votación electrónica predominantes adolecen de deficiencias estructurales que comprometen su integridad y credibilidad. Estudios como los de \citet{garcia2021} identifican variables críticas como: la **centralización de la infraestructura**, que genera puntos únicos de fallo y manipulación; la **opacidad en el registro de votos**, que impide auditorías independientes; y la **vulnerabilidad a ataques cibernéticos**, que amenazan la confidencialidad y disponibilidad del proceso. Estas variables han sido confirmadas en incidentes documentados en diversos países, donde errores de software, fallas en la autenticación y falta de transparencia han minado la confianza pública en los resultados electorales \citep{wef2024}.

En el contexto nacional, aunque existen iniciativas de modernización del Estado y adopción de tecnologías emergentes —como se refleja en la Guía de Implementación de Blockchain del Ministerio TIC \citep{mintic2020}—, no se ha desplegado un sistema integral de votación digital que supere las limitaciones antes mencionadas. El análisis de la situación actual en el ámbito institucional y académico local evidencia una brecha entre el potencial teórico de la tecnología blockchain y su implementación práctica en procesos electorales auditables. Esta brecha se manifiesta en la ausencia de prototipos funcionales que integren de manera coherente la gestión de identidades, la emisión segura de votos y el registro inmutable en una arquitectura accesible y reproducible.

El diagnóstico específico realizado para esta investigación incluyó la revisión de plataformas de votación electrónica existentes (tanto comerciales como académicas), el análisis de reportes de vulnerabilidades en sistemas centralizados y la evaluación de experiencias piloto basadas en blockchain, como las de Sierra Leona \citep{perper2018} y Tsukuba \citep{zhao2018}. Este proceso permitió identificar las siguientes variables clave que estructuran el problema de investigación:
\begin{enumerate}
    \item \textbf{Variable de integridad:} garantía de que cada voto sea único, inmutable y contabilizado correctamente.
    \item \textbf{Variable de privacidad:} protección de la identidad del votante sin comprometer la auditabilidad del proceso.
    \item \textbf{Variable de transparencia:} capacidad de verificación pública y independiente de cada etapa electoral.
    \item \textbf{Variable de escalabilidad:} adaptabilidad del sistema a diferentes volúmenes de votantes y contextos institucionales.
\end{enumerate}

La confrontación de estas variables con el estado actual del objeto de estudio demostró la existencia de un problema científico relevante: la falta de un modelo integral que, aprovechando las propiedades de la blockchain, resuelva simultáneamente las tensiones entre transparencia y privacidad, entre descentralización y eficiencia, y entre seguridad técnica y usabilidad práctica. La pertinencia de esta investigación radica precisamente en proponer y validar un prototipo que aborde estas tensiones, ofreciendo una solución tecnológicamente sólida, metodológicamente reproducible y alineada con los referentes teóricos y prácticos más avanzados en el campo.

En conclusión, el diagnóstico presentado no solo corrobora la veracidad de la situación problemática identificada, sino que también justifica la necesidad de desarrollar una alternativa concreta que supere las limitaciones de los sistemas actuales, contribuyendo así al avance del conocimiento en el área de la gobernanza digital segura y confiable.
}