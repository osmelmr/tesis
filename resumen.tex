\ResumenAbstract
{
Esta investigación aborda el diseño e implementación de una aplicación web de votación digital que integra tecnología blockchain para garantizar integridad, transparencia y verificabilidad del proceso electoral. El estudio surge de la necesidad de superar las limitaciones de los sistemas electrónicos centralizados tradicionales, vulnerables a manipulaciones y con escasa capacidad de auditoría pública. El objetivo general fue desarrollar un prototipo funcional basado en una arquitectura híbrida que combina un backend seguro (Django REST Framework), un frontend intuitivo (React/TypeScript) y una capa de registro inmutable (blockchain simulada). Mediante una metodología iterativa e incremental, se implementaron mecanismos criptográficos para preservar el anonimato del votante mientras se garantizaba la unicidad e inmutabilidad de cada voto. Los resultados obtenidos demuestran que el sistema cumple con los requisitos funcionales establecidos: procesó exitosamente más de 200 votos en pruebas controladas, mantuvo una correspondencia del 100\% entre registros locales y hashes en blockchain, y mostró un tiempo de respuesta promedio de 2.3 segundos por transacción. Las conclusiones confirman la viabilidad técnica de utilizar blockchain como soporte para sistemas electorales digitales auditables, ofreciendo una alternativa confiable para contextos institucionales o académicos que requieran transparencia y seguridad reforzada.
PALABRAS CLAVE\\
votación digital, blockchain, integridad electoral, transparencia, seguridad informática}
{
This research addresses the design and implementation of a web-based digital voting application that integrates blockchain technology to ensure electoral process integrity, transparency, and verifiability. The study arises from the need to overcome limitations of traditional centralized electronic systems, which are vulnerable to manipulation and lack public auditability. The general objective was to develop a functional prototype based on a hybrid architecture combining a secure backend (Django REST Framework), an intuitive frontend (React/TypeScript), and an immutable recording layer (simulated blockchain). Through an iterative and incremental methodology, cryptographic mechanisms were implemented to preserve voter anonymity while guaranteeing the uniqueness and immutability of each vote. The obtained results demonstrate that the system meets the established functional requirements: it successfully processed over 200 votes in controlled tests, maintained 100\% correspondence between local records and blockchain hashes, and showed an average response time of 2.3 seconds per transaction. The conclusions confirm the technical feasibility of using blockchain as a support for auditable digital electoral systems, offering a reliable alternative for institutional or academic contexts requiring enhanced transparency and security.
KEYWORDS\\
digital voting, blockchain, electoral integrity, transparency, cybersecurity}
