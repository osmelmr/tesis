\chapter*{Introducción}
\addcontentsline{toc}{chapter}{Introducción}

La presente tesis aborda el diseño e implementación de una aplicación web de votación digital que integra una capa de registro inmutable basada en tecnología blockchain. El contexto que motiva esta investigación es la necesidad de fortalecer la confianza en los procesos electorales, ya que tanto los sistemas tradicionales como algunas soluciones electrónicas centralizadas han presentado vulnerabilidades que pueden afectar la legitimidad de los resultados y limitar la capacidad de auditoría independiente. La literatura especializada señala que las infraestructuras distribuidas ofrecen trazabilidad e integridad al registro de votos, reduciendo la dependencia de entidades centrales y permitiendo una verificación pública más transparente \citep{garcia2021,wef2024,ibm2023}.

La tecnología blockchain, gracias a su arquitectura criptográfica y a sus mecanismos de consenso, proporciona propiedades de inmutabilidad y verificabilidad particularmente útiles para procesos electorales. No obstante, su incorporación requiere decisiones de diseño que garanticen simultáneamente la privacidad del votante, la unicidad del voto y la integridad de la información registrada. Por ello, esta tesis adopta un enfoque híbrido: la autenticación y autorización se gestionan mediante una API controlada por el backend, mientras que la emisión final del voto se publica como un registro hash en la cadena, manteniendo el anonimato del votante y permitiendo la auditabilidad externa. Este enfoque se fundamenta en recomendaciones técnicas y en estudios previos sobre votación distribuida \citep{nakamoto2008,garcia2021,mintic2020}.

El problema central que aborda este trabajo consiste en asegurar que solo usuarios autorizados puedan emitir voto, que cada voto sea único e inmutable, y que el sistema permita auditorías externas sin revelar identidades. Con este fin, el objetivo principal es construir y validar un prototipo funcional compuesto por un frontend, un backend y una capa de publicación en blockchain. Este prototipo implementa las reglas de negocio básicas del proceso electoral: registro de votantes, gestión de candidaturas, emisión de votos, control de unicidad mediante has\_voted, y publicación del registro hash en la cadena. La propuesta se respalda en experiencias internacionales donde se han probado sistemas de votación basados en blockchain, demostrando su potencial y limitaciones \citep{perper2018,zhao2018,garcia2021}.

El alcance de la tesis se circunscribe a un entorno institucional o académico, sin integrar padrones oficiales ni mecanismos estatales de identificación. Aunque no se incluyen técnicas avanzadas de privacidad, como pruebas de conocimiento cero o cifrado homomórfico, se discuten como propuestas para trabajos futuros. Durante la fase de desarrollo se utilizó una blockchain simulada (mock) con el fin de validar la lógica funcional sin incurrir en los costos operativos de redes públicas; posteriormente se diseñaron los elementos necesarios para migrar a redes de prueba compatibles con la máquina virtual de Ethereum. Esta estrategia permite validar los invariantes esenciales del sistema —unicidad del voto, integridad del hash y trazabilidad del proceso— sin comprometer la reproducibilidad experimental \citep{mintic2020,wef2024}.

En cuanto a la metodología, se adoptó un enfoque iterativo e incremental que incluyó la identificación de requisitos a partir de la lógica de negocio del proceso electoral, el modelado de datos, el diseño arquitectónico, el desarrollo modular de la API REST y la interfaz de usuario, y la simulación de la capa blockchain. Asimismo, se realizaron pruebas unitarias, de integración y de sistema con el fin de verificar la correspondencia entre los registros locales y las transacciones simuladas, así como de evaluar escenarios de falla como intentos de voto duplicado o emisión fuera de la ventana temporal establecida. Este proceso metodológico contribuye a la construcción de un sistema robusto fundamentado en principios de seguridad y confiabilidad \citep{ibm2023,garcia2021}.

Finalmente, esta tesis presenta como resultados: (a) un modelo de datos formalizado y reglas de negocio que estructuran el proceso electoral digital; (b) un prototipo funcional capaz de ejecutar el flujo completo desde la autenticación hasta la publicación del hash del voto; y (c) un conjunto de pruebas que permiten evaluar la integridad, la usabilidad y el comportamiento del sistema bajo diferentes condiciones. La organización del documento refleja este recorrido conceptual y técnico, desde el marco teórico hasta las conclusiones, y está orientada a facilitar la ampliación de este trabajo en contextos académicos o institucionales que busquen explorar el uso de tecnologías distribuidas para procesos electorales \citep{garcia2021,wef2024}.
